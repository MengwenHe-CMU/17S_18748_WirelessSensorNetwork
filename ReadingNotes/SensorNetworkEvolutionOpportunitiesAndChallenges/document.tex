\documentclass[letterpaper,10pt]{article}
\usepackage[margin=2cm]{geometry}

\usepackage{graphicx}
\usepackage{amsmath}
\usepackage{amsfonts}
\usepackage{amssymb}
\usepackage[colorlinks]{hyperref}

\newcommand{\panhline}{\begin{center}\rule{\textwidth}{1pt}\end{center}}

\title{\textbf{Connecting the Physical World with Pervasive Networks}}
\author{CHEE-YEE CHONG, SRIKANTA P. KUMAR}

\begin{document}

\maketitle

\panhline
\href{../index.html}{Back to Index}

\panhline
\tableofcontents

\section*{Resources}

\begin{itemize}
	\item \href{../../Readings/Sensor Networks Evolution, Opportunities, and Challenges.pdf}{Paper}
\end{itemize}

\panhline

\section{Introduction}

\begin{itemize}
	\item Networked microsensors technology is a key technology for the future.
	\item Smart disposable microsensors can be deployed almost anywhere.
	\item Ubiquitous wireless networks of microsensors probably offer the most potential in changing the world of sensing.
	\item Paper structure:
	\begin{itemize}
		\item History of research in sensor networks
		\item Technology trends
		\item New applications
		\item Research issues and hard problems
		\item Some examples
	\end{itemize}
\end{itemize}
	
\section{History of Research in Sensor Networks}

\begin{itemize}
	\item The devbelopment of sensor networks requires technologies from three different research areas:
	\begin{itemize}
		\item Sensing
		\item Communication
		\item Computing (hardware, software, and algorithms)
	\end{itemize}
\end{itemize}

\subsection{Early Research on Military Sensor Networks}

\begin{itemize}
	\item Can be traced back to Cold War.
	\item The sensor networks generally adopt a \textbf{hierarchical} processing structure where processing occurs at consecutive levels until the information about events of interest reaches the user.
\end{itemize}

\subsection{Distributed Sensor Networks (DSN) Program at the Defence Advanced Research Projects Agency (DARPA)}

\begin{itemize}
	\item Modern research on sensor networks started around 1980 with the DSN program at the DARPA.
	\item R. Kahn, who was coinventor of the TCP/IP protocols and played a key role in developing the Internet, wanted to know whether the Arpanet (predecessor of the Internet) approach for communication could be extended to sensor networks.
	\item The network was assumed to havbe many spatially distributed low-cost sensing nodes that collaborate with each other but operate autonomously, with information being routed to whichever node can best use the information.
	\item Researchers:
	\begin{itemize}
		\item CMU focused on providing a network operating system that allows flexible, transparent access to distributed resources needed for a fault-tolerant DSN. They developed a communication-oriented operating system called Accent (evolved into the Mach OS), whose primitives support transparent networking, system reconfiguration, and rebinding.
		\item MIT focused on knowledge-based signal processing techniques for tracking helicopters using a distributed array of acoustic microphones by means of signal abstractions and matching techniques.
		\item ADS\footnote{Advanced Decision Systems, Mountain View, CA} developed a multiple-hypothesis tracking algorithm to deal with difficult situations involving high target density, missing detections, and false alarms, and decomposed the algorithm for distributed implementation.
	\end{itemize}
\end{itemize}

\subsection{Military Sensor Networks in the 1980s and 1990s}

\begin{itemize}
	\item In platform-centric warfare: platforms own specific weapons,k which in turn own sensors in a fairly rigid architecture.
	\item In network-centric warfare: sensors donot necessarily belong to weapons or platforms. Instead, they collaborate with each other over a communication network, and information is sent to the appropriate shooters.
\end{itemize}

\subsection{Sensor Network Research in the 21st Century}

\begin{itemize}
	\item Latest technological advances boosted sensor networks development
	\item Sensor Information Technology (SensIT) program pursued two key research and development thrusts:
	\begin{itemize}
		\item New networking techniques: sensor devices or nodes should e ready for rapid deployment, in an \textit{ad hoc} fasion, and in highly dynamic environments.
		\item NEtworked information processing.
	\end{itemize}
\end{itemize}

\section{Technology Trends}

\begin{itemize}
	\item Sensors, processors, and communication devices are all getting much smaller and cheaper.
	\item WIreless netowrkds based upon IEEE 802.11 standards can now provide bandwidth approaching those of wired networks.
	\item IEEE 802.15 standard for personal area networks (PANs) have a radius of 5 to 10 meters.
	\item In the future, the advances in MEMS technology will produce sensors that are even more capable and versatile.
\end{itemize}

\section{New Applications}

Examples:
\begin{itemize}
	\item Infrastructure security
	\item Environment and Habitat Monitoring
	\item Industrial Sensing
	\item Traffic Control
\end{itemize}


\section{Hard Problems And Technical Challenges}

\subsection{Ad Hoc Network Discovery}

\begin{itemize}
	\item Knowledge of the network is essential for a sensor in the network to operate properly.
	\item In planned networks, the topology of the network is usually known a priori.
	\item For ad hoc networks, the network topology has to be constructed in real time, and updated periodically as sensors fail or new sensors are deployed.
	\item In addition to knowledge of the topology, each sensor also needs to know its own location.
\end{itemize}

\subsection{Network Control and Routing}

\begin{itemize}
	\item The network muse deal with resources that are dynamically changing, and the system should operate autonomously, changing its configuration as required.
	\item Without requiring IP addresses at each node is that one can deploy network devices in very large numbers.
	\item Survivability and adaptation to the environment are ensured through deploying an adequate number of nodes to provide redundancy in paths, and algorithms to find the right paths.
\end{itemize}

\subsection{Collaborative Signal and Information Processing}

\begin{itemize}
	\item Collaborative signal and information processing over a network is a new area of research and is related to distributed information fusion.
	\begin{itemize}
		\item Issue1: degree of information sharing between nodes and how nodes fuse the information from other nodes.
		\item Issue2: tradeoff between performance and robustness.
		\item Issue3: optimal data association is computationally expensive and requires significant bandwidth for communication.
		\item meet mission latency and reliability requirements.
		\item how to maximize sensor network operational life.
	\end{itemize}
\end{itemize}

\subsection{Tasking and Querying}

\begin{itemize}
	\item A sensor field is like a database with many unique features, and it is challenging for querying because of its instability.
	\item It is important that users have a simple interface to interactively task and query the sensor network.
	\item Mobile platforms can carry sensors and query devices. As a result, seamless internetwork between mobile and fixed devbices in the absence of any infrastructure is a critical and unique requirement for sensor networks.
\end{itemize}

\subsection{Security}

Sensor network should be protected against intrusion and spoofing.

\section{Some Recent Results}

\subsection{Localized Algorithms and Directed Diffusion}

\begin{itemize}
	\item Centralized algorithms:
	\begin{itemize}
		\item collect data from multiple sensor nodes
		\item potentially provide the best performance
		\item undesireable because of high communication cost and lack of robustness and reliability.
	\end{itemize}
	\item Localized algorithms:
	\begin{itemize}
		\item the sensor nodes only communicate with sensors within a neighborhood.
		\item robust to network changes and node failures.
		\item scale will with increasing network size.
		\item Difficult to design because of the potentially complicated relationship between local behavior and global behavior.
	\end{itemize}
	\item Estrin et al. developed directed diffusion routing algorithms that belong to the class of localized algorithms.
	\begin{itemize}
		\item Diffusion is a form of broadcast routing that does not specify a destination node address.
		\item Packets are forwarded to neighboring nodes, and a direction or gradient is overlaid to control the broadcast or forwarding of the packet, which eventually reaches the destination.
		\item The gradient could be based on geographic information or other attributes such as power, congestion and other resources available in the network nodes.
		\item It requires considerably less energy than standard routing mechanisms such as flooding and omniscient multicast.
	\end{itemize}
\end{itemize}

\subsection{Distributed Tracking in Wireless Ad Hoc Networks}

\begin{itemize}
	\item Tracking in ad hoc networks with microsensors poses different challenges due to communication, processing and energy constraints.
	\item Zhao et al. addressed the dynamic sensor collaboration problem in distributed tracking to determine dynamically which sensor is most appropriate to perform the sensing, what needs to be sensed, and to whom to communicate the information.
	\item Distributed data association algorithms are available for networks with large nodes but are computationally too expensive to implement on ad hoc networks. An approximate approach for cheap data association is called identity management.
\end{itemize}

\subsection{Distributed Classification in Sensor Networks Using Mobile Agents}

\begin{itemize}
	\item The bandwidth of a wireless sensor network is typically lower than that of a wired network, a sensor network's communications requirements may exceed their capacities.
	\item Mobile agents have been proposed as a solution to this dilemma (above sentence).
	\item In a mobile-agent-based DSN, data stay at each local site or sensor, while the integration or fusion code is moved to the data. Communication bandwidth requrieemnt may be reduced if the agent is smaller in size than the data.
\end{itemize}

\end{document}



