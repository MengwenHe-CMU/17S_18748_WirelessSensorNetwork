\documentclass[letterpaper,10pt]{article}
\usepackage[margin=2cm]{geometry}

\usepackage{graphicx}
\usepackage{amsmath}
\usepackage{amsfonts}
\usepackage{amssymb}
\usepackage[colorlinks]{hyperref}

\newcommand{\panhline}{\begin{center}\rule{\textwidth}{1pt}\end{center}}

\title{\textbf{Connecting the Physical World with Pervasive Networks}}
\author{Deborah Estrin, David Culler and Kris Pister, Gaurav Sukhatme}

\begin{document}

\maketitle

\panhline
\href{../index.html}{Back to Index}

\panhline
\tableofcontents

\section*{Resources}

\begin{itemize}
	\item \href{../../Readings/Connecting the Physical World with Pervasive Networks.pdf}{Paper}
\end{itemize}

\panhline

\section{Challenges}

The most serious impediments to pervasive computing's advances are systems challenges:
\begin{itemize}
	\item immense amount of distributed system elements,
	\item limited physical access to them,
	\item and this regime's extreme environmental dynamics
\end{itemize}

\subsection{Immense Scale}

Dense instrumentation of complex physical systems $\Rightarrow$ devices must scale down to extremely small volume, with applications formulated in terms of immense numbers of them.

\subsection{Limited Access}

Many devices will be embedded in the environment in places that are inaccessible or expensive to connect with wires, making the individual system elements largely untethered, unattended, and resource constrained.

\subsection{Extreme Dynmaics}

By virtue of nodes and the system as a whole being closely tied to the ever-changing physical world, these systems will experience extreme dynamics.

\section{A Taxonomy of Systems}

\begin{itemize}
	\item Scale: the finer grain the sampling, the more important the innovative collaborative signal processing techniques such as those described elsewhere.
	\begin{itemize}
		\item Sampling: The physical phenomena measured ultimately dictates spatial and temporal sampling scale.
		\item Extend: The spatial and temporal extent of systems also varies widely.
		\item Density: System density is a measure of sensor nodes per footprint of input stimuli.
	\end{itemize}
	\item Variability: Relatively static systems emphasize design time optimization whereas more variable systems must must runtime self-organization and might be fundamentally limited in the extent to which they are both variable and long-lived.
	\begin{itemize}
		\item Structure: Ad hoc versus engineered system structure refers to the variability in system composition.
		\item Task: Variability in system task determines the extent to which we can optimize the system of a single mode of operation.
		\item Space: Variability in space--meaning mobility--applies to both system nodes and phenomena.
	\end{itemize}
	\item Autonomy: The degree of autonomy has some of the most significant and varied long-term consequences for system design: the higher the overall system's autonomy, the higher the overall system's autonomy, the less the human involvement and the greater the need for extensive and sophisticated processing inside the system.
	\begin{itemize}
		\item Modalities: Different modalities provide noise resilience to on another and can combine to eliminate noise and identify anomalous measurements.
		\item Complexity: Greater system autonomy also entails greater complexity in the computational model.
	\end{itemize}
\end{itemize}

\section{Where are we now?}

In each of the following areas, we see a consistent trend from highly engineered deployments of modest scale using application-specific devices to ad hoc deployments of immense scale based on reusable components intended for system evolution.

\subsection{Small Packages in the Physical World}

\subsection{Sensing and Actuation}

\subsection{Localization}

\subsection{A Distributed System Architecture}

\section{Where are we headed?}



\end{document}



